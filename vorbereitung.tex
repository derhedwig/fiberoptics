\documentclass[a4paper]{scrartcl}

\usepackage[utf8]{inputenc}
\usepackage[T1]{fontenc}
\usepackage{lmodern}
\usepackage[ngerman]{babel}
\usepackage{amsmath}

\title{Vorbereitungsaufgaben zum Praktikum „Optische Informationsübertragung“}
\author{Henrik Kröger \and Nicolas Polido}
\date{\today}
\begin{document}

\maketitle
%\tableofcontents

\subsection*{Definitionen}

Normalisierte Frequenz:
\begin{equation}
    V = k_0 a \text{NA} = \frac{2\pi}{\lambda_0} a \sqrt{n_\text{K}^2 - n_\text{M}^2}
\end{equation}

Brechzahl-Unterschied zwischen Mantel und Kern:
\begin{equation}
    \Delta = \frac{n_\text{K}^2 - n_\text{M}^2}{2n_\text{K}^2}
\end{equation}


\subsection*{Erläutern Sie den Unterschied zwischen einer Einmoden- und Vielmodenfaser
(Anzahl der Moden und Dispersion)}

Eine Einmodenfaser ist so klein bzw. so gestaltet, dass sie
(für eine gegebene Wellenlänge $\lambda$),
nur eine Mode (die $\text{LP}_{01}$-Grundmode) zulässt.
Eine Vielmodenfaser dagegen lässt viele Moden zu.
In allen Fasern gibt es chromatische Dispersion, aber in Einmodenfasern gibt es
keine \emph{modale Dispersion} – höhere Moden bewegen sich langsamer.


\subsection*{Wie sehen die Brechzahlprofile einer polarisationserhaltenden sowie einer
biegeunempfindlichen Einmodenfaser aus?}

\begin{itemize}
    \item Polarisationserhaltend -> ecliptic or PANDA, bowtie
    \item biegeunempfindlich -> W fiber
\end{itemize}


\subsection*{Welche Arten von Streuung treten in optischen Glasfasern auf?}
\begin{itemize}
    \item lineare Streuung
    \begin{itemize}
        \item Rayleigh-Streuuung\\
              Kommt durch mikroskopische Abweichungen im Material
              (Glas besteht aus verschiedenen Oxiden
               $\text{Si}\text{O}_2$, $\text{P}_2\text{O}_5$, \ldots)
        \item Mie-Streuung\\
              \dots
        \item Wellenleiter-Streuung\\
              Abweichungen des Kerndurchmessers oder des Kern-Mantel-Übergangs
              oder Abweichungen von $n$ im Kern oder im Mantel
    \end{itemize}

    \item nicht-lineare Streuung
    \begin{itemize}
        \item stimulierte Brillouin-Streuung\\
              Kommt durch thermische Molekularvibrationen und generiert akustische Phononen.
        \item stimulierte Raman-Streuung\\
              Generiert optische Phononen.

    \end{itemize}
\end{itemize}


\subsection*{Was ist ein Puls-OTDR und wie funktioniert es?
Wie berechnet sich die Ortsauflösung eines OTDRs?}
OTDR heißt \emph{Optical Time Domain Reflectrometry},
zu deutsch Optische Zeitbereichsreflektometrie.
Bei OTDR wird ein kurzer Laserpuls in den Fiber geschickt.
Licht, das wieder zurück kommt, wird gemessen.

Die \emph{Ortsauflösung} hängt von der Breite des Laserpulses ab.
Nach dem Rayleigh-Kriterium sind zwei Pulse unterscheidbar,
wenn der eine im Minimum des anderen liegt.
Daraus ergibt sich, dass die Ortsauflösung der halben Pulsdauer entspricht:
\begin{equation*}
    \frac{\Delta \tau}{2} \cdot n c
\end{equation*}


\subsection*{Was sind FBGs und wie werden diese hergestellt? Wo werden FBGs in der OKT eingesetzt?}
\emph{Faser-Bragg-Gitter} sind in Lichtwellenleiter mit UV-Licht eingeschriebene optische Interferenzfilter.
Sie reflektieren Licht einer bestimmten Wellenlänge zurück:
\begin{equation*}
    \lambda_B = n_\text{eff} \cdot 2 \Lambda = n_\text{eff} \cdot \lambda
\end{equation*}




\subsection*{Erklären Sie die Funktionsweise eines Mach-Zehnder (MZ)-Interferometers}



\subsection*{Wie wird die Kohärenzlänge einer Lichtquelle berechnet?}
Nach dem Wiener-Chintschin-Theorem berechnet sich die Kohärenzlänge:
\begin{equation*}
     l_c=\frac{\lambda^2}{\Delta\lambda}
\end{equation*}


\subsection*{Beschreiben Sie mathematisch das Prinzip eines Direkten- sowie eines Homodynempfängers}
???

\subsection*{Welche Multiplextechniken existieren für die OKT?}

\subsection*{Welche Modulationsformate existieren für die OKT?}

\subsection*{Bitte beschreiben Sie die Funktionsweise eines Abtastoszilloskops}

\subsection*{Was ist ein Augendiagramm und welche Aussagen lässt es zu?}

\end{document}
