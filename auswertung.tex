\documentclass[a4paper]{article}

\usepackage[utf8]{inputenc}
\usepackage[T1]{fontenc}
\usepackage{lmodern}
\usepackage[ngerman]{babel}
\usepackage{amsmath}
%\usepackage[amssymb]{SIunits}
%\usepackage{sistyle}
%\SIstyle{German}
\usepackage{hyperref}
\usepackage[square]{natbib}


\title{Versuche zur Optischen Informationsübertragung}
\author{Henrik Kröger \and Nicolás Pulido}
\date{Durchgeführt am 26. und 27. April 2016}

\begin{document}
\bibliographystyle{geralpha}

\maketitle

%\noindent Einleitungstext

\tableofcontents

\newpage
\section{Faseroptische Übertragungsstrecke}

\subsection{lineare Degradationen}

\subsubsection{Verluste}
\paragraph{Streuung}
\paragraph{Dämpfung}
\paragraph{Koppelverluste}
\paragraph{Krümmungsverluste}
\ \\

\noindent Wenn Verluste wegen Unreinigungen fast ganz beseitigt sind, es
bleiben noch die Verluste, die wegen die eigentliche Struktur des Glases
verursacht sind.  Wenn der Krümmungsradius im Bereich der Zentimeter liegt,
spricht man von makroskopischen Krümmungsverluste. 

In \emph{Multimodenfaser} kann man die strahlenoptische Beschreibung benutzen.
Bei eine Krümmung kann der kritische Winkel für die Totalreflexion
überschritten werden in der gebogenen Portion. 

Man musst für Multimodenfaser die wellenoptische Beschreibung benutzen. Die
Feldverteilung jeder Mode erstreckt sich auch in das Mantel. In der gekrümmten
Stelle des Fasers gibt es eine Stelle, gegen dem äußerem Teil der Kurve, wo die
Propagationsgeschwindigkeit fängt zu überschreiten, die im Medium durch den
Brechungsindex erlaubte Geschwindigkeit. Der Wellenfront wird effektiv nicht
mehr eben und daher zeigt ein Komponent des Poynting-Vektors nach außen, in
senkrechter Richtung zur Faser, was zur Energieverluste durch Radiation führt. 

Das Faser wird mechanisch am äußeren Teil der Kurve expandiert und am inneren
Teil komprimiert, was der Brechungsindex in dieser Regionen so verändert, dass 
die Verluste einigermaßen zu kompensiert werden, obwohl dies nicht genug ist um
die Verluste zu vermeiden.

Es ist klar: je weniger dringt der Feld im Mantel ein, desto weniger
Krümmungsverluste gibt es. 


\subsubsection{Chromatische Dispersion}
Ein Signal kann sich verzerren, während es sich durch einen optischen
Träger fortbewegt. Viele Arten dieser Verzerrungen passieren, weil
verschiedene Teile des Signals sich mit verschiedenen Geschwindigkeiten
bewegen. Es besteht die Gefahr, dass das Signal beim Empfänger so
stark verzerrt ankommt, dass er nicht mehr in der Lage ist, es zu
entschlüsseln. In \emph{Multimodenfasern} findet man:
\begin{description}
  \item[Modendispersion:] Jede Mode kann unterschiedlich
    tief in den Mantel der Faser eindringen und sich damit mit verschiedenen
    Geschwindigkeiten fortpflanzen.
\end{description} In \emph{Monomodenfaser} gibt es auch Dispersion. Der Grund
dafür ist, dass der Brechungsindex abhängig von der Wellenlänge ist.
Lichtpulse oder Wellenpakete haben immer eine gewisse spektrale Breite und die
verschiedenen Frequenzkomponenten des Pulses pflanzen sich dann mit
unterschiedlichen Geschwindigkeiten fort, was zu Verzerrungen des Pulses führt.
Man kann im Monomodenfaser folgende Dispersionsarten unterscheiden:
\begin{description}
  \item[Materialdispersion:] Dies ist eine direkte Konsequenz der
    Abhängigkeit des Brechungsindexes von der  Wellenlänge und es geschieht
    nicht nur in optischen Fasern, sondern auch in jedem Glas und ist unabhängig von
    der Geometrie der Faser.
  \item[Wellenleiterdispersion:] Der Lichtpuls propagiert auch im Mantel,
    der einen anderen Brechungsindex als der Kern hat. Wenn man die
    Wellenlänge variiert, gibt es einen Übergang zwischen Propagation,
    die meist im Kern und Propagation, die meist im Mantel stattfindet.
    Dies macht den effektiven Brechungsindex abhängig von der Wellenlänge.
  \item[Profildispersion:] Sie ist kleiner als die Moden- und
    Materialdispersion. Der Unterschied zwischen den Brechungsindexen von Kern
    und Mantel ist auch von der Wellenlänge abhängig.
  \item[Polarisationsmodendispersion:] Sie ist auch klein. Die Polarisation des
    Lichtes kann in zwei orthogonale Komponenten zerlegt werden. Der
    Brechungsindex von Glas ist streng genommen nicht ganz kreissymmetrisch,
    d.h. jedes Glas ist ein bisschen doppelbrechend. Verschiedene
    Brechungsindizes für jede Polarisationskomponente führen zur Dispersion.
\end{description}

\subsection{nicht-lineare Degradationen}
\paragraph{Vierwellenmischung}
\paragraph{Selbstphasenmodulation}
\paragraph{Kreuzphasenmodulation}

\subsection{Versuch}
Es werden Übertragungsverluste an einem optischen Übertragungssystem
mithilfe von \emph{Optischer Zeitbereichsreflektometrie (OTDR)} und
einem Powermeter untersucht.
Die \emph{Rückstreukurve} soll durch unterschiedliche Fasertypen etc.
manipuliert werden.


\newpage
\section{Komponenten eines faseroptischen Übertragungssystems}
\subsection{Komponenten}
\paragraph{Sendeseite}
\paragraph{Empfängerseite}
\paragraph{Modulator}
\paragraph{Koppler}
\paragraph{optische Verstärker}
\paragraph{Polarisationssteller}
\paragraph{Isolator}
\paragraph{Zirkulator}
\paragraph{Faser-Bragg-Gitter}

\subsection{Versuch}
\subsubsection{Spektren}
\subsubsection{Signalübertragungsverhalten Mach-Zehnder-Modulator}
\subsubsection{Herstellung und Untersuchung Faser-Bragg-Gitter}


\newpage
\section{Systemdesign sowie Multiplex und Modulationsformate}
\subsection{unterschiedliche Multiplextechniken}
\subsection{unterschiedliche Modulationsformate}
\subsection{Parameter für Systemdesign}
\paragraph{Leitungsbudget}
\paragraph{Bitfehlerrate}
\paragraph{Jitter}
\paragraph{Augendiagramm}
\paragraph{Signal-zu-Rausch-Verhältnis}
\paragraph{Empfängerempfindlichkeit}

\subsection{Versuch}
Charakterisierung der optischen Übertragungsstrecke:
Bitfehlerrate in Abhängigkeit von der Dämpfung,
Augendiagramm in Abhängigkeit von der Dämpfung.

\end{document}
