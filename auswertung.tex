\documentclass[a4paper]{article}

\usepackage[utf8]{inputenc}
\usepackage[T1]{fontenc}
\usepackage{lmodern}
\usepackage[ngerman]{babel}
\usepackage{amsmath}
%\usepackage[amssymb]{SIunits}
%\usepackage{sistyle}
%\SIstyle{German}
\usepackage{hyperref}
\usepackage[square]{natbib}


\title{Versuche zur Optischen Informationsübertragung}
\author{Henrik Kröger \and Nicol\'as Pulido}
\date{Durchgeführt am 26. und 27. April 2016}

\begin{document}
\bibliographystyle{geralpha}

\maketitle

%\noindent Einleitungstext

\tableofcontents

\newpage
\section{Faseroptische Übertragungsstrecke}

\subsection{lineare Degradationen}

\subsubsection{Verluste}
\paragraph{Streuung}
\paragraph{Dämpfung}
\paragraph{Koppelverluste}
\paragraph{Krümmungsverluste}

\subsubsection{Dispersion}
\paragraph{Chromatische-Dispersion}
\paragraph{Moden-Dispersion}
\paragraph{Polarisationsmodendispersion}
\paragraph{Materialdispersion}

\subsection{nicht-lineare Degradationen}
\paragraph{Vierwellenmischung}
\paragraph{Selbstphasenmoduldation}
\paragraph{Kreuzphasenmodulation}

\subsection{Versuch}
Es werden Übertragungsverluste an einem optischen Übertragungssystem
mithilfe von \emph{Optischer Zeitbereichsreflektometrie (OTDR)} und
einem Powermeter untersucht.
Die \emph{Rückstreukurve} soll durch unterschiedliche Fasertypen etc.
manipuliert werden.


\newpage
\section{Komponenten eines faseroptischen Übertragungssystems}
\subsection{Komponenten}
\paragraph{Sendeseite}
\paragraph{Empfängerseite}
\paragraph{Modulator}
\paragraph{Koppler}
\paragraph{optische Verstärker}
\paragraph{Polarisationssteller}
\paragraph{Isolator}
\paragraph{Zirkulator}
\paragraph{Faser-Bragg-Gitter}

\subsection{Versuch}
\subsubsection{Spektren}
\subsubsection{Signalübertragungsverhalten Mach-Zehnder-Modulator}
\subsubsection{Herstellung und Untersuchung Faser-Bragg-Gitter}


\newpage
\section{Systemdesign sowie Multiplex und Modulationsformate}
\subsection{unterschiedliche Multiplextechniken}
\subsection{unterschiedliche Modulationsformate}
\subsection{Parameter für Systemdesign}
\paragraph{Leitungsbudget}
\paragraph{Bitfehlerrate}
\paragraph{Jitter}
\paragraph{Augendiagramm}
\paragraph{Signal-zu-Rausch-Verhältnis}
\paragraph{Empfängerempfindlichkeit}

\subsection{Versuch}
Charakterisierung der optischen Übertragungsstrecke:
Bitfehlerrate in Abhängigkeit von der Dämpfung,
Augendiagramm in Abhängigkeit von der Dämpfung.

\end{document}
