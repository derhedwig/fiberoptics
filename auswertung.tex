\documentclass[a4paper]{article}

\usepackage[utf8]{inputenc}
\usepackage[T1]{fontenc}
\usepackage{lmodern}
\usepackage[ngerman]{babel}
\usepackage{amsmath}
%\usepackage[amssymb]{SIunits}
%\usepackage{sistyle}
%\SIstyle{German}
\usepackage{hyperref}
\usepackage[square]{natbib}


\title{Versuche zur Optischen Informationsübertragung}
\author{Henrik Kröger \and Nicol\'as Pulido}
\date{Durchgeführt am 26. und 27. April 2016}

\begin{document}
\bibliographystyle{geralpha}

\maketitle

%\noindent Einleitungstext

\tableofcontents

\newpage
\section{Faseroptische Übertragungsstrecke}

\subsection{lineare Degradationen}

\subsubsection{Verluste}
\paragraph{Streuung}
\paragraph{Dämpfung}
\paragraph{Koppelverluste}
\paragraph{Krümmungsverluste}

\subsubsection{Chromatische Dispersion}
Ein Signal kann sich verzerren, w\"ahrend es sich durch ein optisches
Tr\"ager fortbewegt. Viele Arten von diese Verzerrungen passieren, weil
verschiedene Teile des Signals sich mit verschiedene Geschwindigkeiten
bewegen. Es besteht das Gefahr, dass der Signal beim Empf\"anger so
stark verzerrt ankommt, dass er nicht mehr in der Lage ist, es zu
entschl\"usseln. In \emph{Multimodenfasern} findet Statt die 
\begin{description}
  \item[Modendispersion:] Jede Mode kann unterschiedlich
    viel im Mantel des Fasers eindringen und damit sich mit verschiedenen
    Geschwindigkeiten fortpflanzen.
\end{description} In \emph{Monomodenfaser} gibt es auch Dispersion. Der Grund
daf\"ur ist, dass der Brechungsindex abh\"angig von der Wellenl\"ange ist.
Lichtpulse oder Wellenpackete haben immer eine gewisse spektrale Breite und die
verschiedene Frequenzkomponenten des Pulses pflanzen sich dann mit
unterschiedliche Geschwindigkeiten fort, was zu Verzerrungen des Pulses kommt.
Man kann im Monomodenfaser folgende Dispersionarten unterscheiden:
\begin{description}
  \item[Materialdispersion:] Dies ist eine direkte Konsequenz der
    Abh\"angigkeit der Brechungsindex von der  Wellenl\"ange und es gescheht
    nicht nur in optischen Fasern aber auch in jedes Glass und ist unabh\"angig von
    der Geometrie des Fasers.
  \item[Wellenleiterdispersion:] Der Lichtpuls propagiert sich auch im Mantel,
    der ein verschiedenen Brechungsindex als der Kern hat. Wenn man die
    Wellenl\"ange variiert, bekommt man ein \"Ubergang zwischen Propagation
    meistens im Kern und Propagation meistens im Mantel. Dies macht der
    effektiver Brechungsindex abh\"angig von der Wellenl\"ange.
  \item[Profildispersion:] Sie ist kleiner als die Moden- und
    Materialdispersion. Die Unterschied zwischen den Brechungsindexen vom Kern
    und Mantel ist auch von der Wellenl\"ange abh\"angig.
  \item[Polarisationsmodendispersion:] Sie ist auch klein. Die Polarisation des
    Lichtes kann in zwei orthogonale Komponenten zerlegt werden. Der
    Brechungsindex von Glass ist steng genommen nicht ganz kreissymetrisch,
    d.h. jeder Glass ist ein bisschen doppelbrechend. Verschiedene
  Brechungsindices f\"ur jede Polarisationskomponente f\"uhrt zur Dispersion.
\end{description}

\subsection{nicht-lineare Degradationen}
\paragraph{Vierwellenmischung}
\paragraph{Selbstphasenmoduldation}
\paragraph{Kreuzphasenmodulation}

\subsection{Versuch}
Es werden Übertragungsverluste an einem optischen Übertragungssystem
mithilfe von \emph{Optischer Zeitbereichsreflektometrie (OTDR)} und
einem Powermeter untersucht.
Die \emph{Rückstreukurve} soll durch unterschiedliche Fasertypen etc.
manipuliert werden.


\newpage
\section{Komponenten eines faseroptischen Übertragungssystems}
\subsection{Komponenten}
\paragraph{Sendeseite}
\paragraph{Empfängerseite}
\paragraph{Modulator}
\paragraph{Koppler}
\paragraph{optische Verstärker}
\paragraph{Polarisationssteller}
\paragraph{Isolator}
\paragraph{Zirkulator}
\paragraph{Faser-Bragg-Gitter}

\subsection{Versuch}
\subsubsection{Spektren}
\subsubsection{Signalübertragungsverhalten Mach-Zehnder-Modulator}
\subsubsection{Herstellung und Untersuchung Faser-Bragg-Gitter}


\newpage
\section{Systemdesign sowie Multiplex und Modulationsformate}
\subsection{unterschiedliche Multiplextechniken}
\subsection{unterschiedliche Modulationsformate}
\subsection{Parameter für Systemdesign}
\paragraph{Leitungsbudget}
\paragraph{Bitfehlerrate}
\paragraph{Jitter}
\paragraph{Augendiagramm}
\paragraph{Signal-zu-Rausch-Verhältnis}
\paragraph{Empfängerempfindlichkeit}

\subsection{Versuch}
Charakterisierung der optischen Übertragungsstrecke:
Bitfehlerrate in Abhängigkeit von der Dämpfung,
Augendiagramm in Abhängigkeit von der Dämpfung.

\end{document}
