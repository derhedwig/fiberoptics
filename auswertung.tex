\documentclass[a4paper]{article}

\usepackage[utf8]{inputenc}
\usepackage[T1]{fontenc}
\usepackage{lmodern}
\usepackage[ngerman]{babel}
\usepackage{amsmath}
%\usepackage[amssymb]{SIunits}
%\usepackage{sistyle}
%\SIstyle{German}
\usepackage{hyperref}
\usepackage[square]{natbib}
\usepackage{upgreek}
\usepackage{tikz}
\usepackage{pgfplots}
\pgfplotsset{width=11cm, compat=1.11}
%\pgfplotsset{compat=1.11}
\usepackage{float} % to use the [H] parameter in figures
\usepackage[margin=1cm, font=small, labelfont=bf]{caption} % caption margin


\title{Versuche zur Optischen Informationsübertragung}
\author{Henrik Kröger \and Nicolás Pulido}
\date{Durchgeführt am 26. und 27. April 2016}

\begin{document}
\bibliographystyle{geralpha}

\maketitle

%\noindent Einleitungstext

\tableofcontents

\newpage
\section{Faseroptische Übertragungsstrecke}

\subsection{lineare Degradationen}

\subsubsection{Verluste}
\paragraph{Streuung}
\paragraph{Dämpfung}
\paragraph{Koppelverluste}
\paragraph{Krümmungsverluste}
\ \\

\noindent Wenn Verluste wegen Unreinigungen fast ganz beseitigt sind, es
bleiben noch die Verluste, die wegen die eigentliche Struktur des Glases
verursacht sind.  Wenn der Krümmungsradius im Bereich der Zentimeter liegt,
spricht man von makroskopischen Krümmungsverluste. 

In \emph{Multimodenfaser} kann man die strahlenoptische Beschreibung benutzen.
Bei eine Krümmung kann der kritische Winkel für die Totalreflexion
überschritten werden in der gebogenen Portion. 

Man musst für Multimodenfaser die wellenoptische Beschreibung benutzen. Die
Feldverteilung jeder Mode erstreckt sich auch in das Mantel. In der gekrümmten
Stelle des Fasers gibt es eine Stelle, gegen dem äußerem Teil der Kurve, wo die
Propagationsgeschwindigkeit fängt zu überschreiten, die im Medium durch den
Brechungsindex erlaubte Geschwindigkeit. Der Wellenfront wird effektiv nicht
mehr eben und daher zeigt ein Komponent des Poynting-Vektors nach außen, in
senkrechter Richtung zur Faser, was zur Energieverluste durch Radiation führt. 

Das Faser wird mechanisch am äußeren Teil der Kurve expandiert und am inneren
Teil komprimiert, was der Brechungsindex in dieser Regionen so verändert, dass 
die Verluste einigermaßen zu kompensiert werden, obwohl dies nicht genug ist um
die Verluste zu vermeiden.

Es ist klar: je weniger dringt der Feld im Mantel ein, desto weniger
Krümmungsverluste gibt es. 


\subsubsection{Chromatische Dispersion}
Ein Signal kann sich verzerren, während es sich durch einen optischen
Träger fortbewegt. Viele Arten dieser Verzerrungen passieren, weil
verschiedene Teile des Signals sich mit verschiedenen Geschwindigkeiten
bewegen. Es besteht die Gefahr, dass das Signal beim Empfänger so
stark verzerrt ankommt, dass er nicht mehr in der Lage ist, es zu
entschlüsseln. In \emph{Multimodenfasern} findet man:
\begin{description}
  \item[Modendispersion:] Jede Mode kann unterschiedlich
    tief in den Mantel der Faser eindringen und sich damit mit verschiedenen
    Geschwindigkeiten fortpflanzen.
\end{description} In \emph{Monomodenfaser} gibt es auch Dispersion. Der Grund
dafür ist, dass der Brechungsindex abhängig von der Wellenlänge ist.
Lichtpulse oder Wellenpakete haben immer eine gewisse spektrale Breite und die
verschiedenen Frequenzkomponenten des Pulses pflanzen sich dann mit
unterschiedlichen Geschwindigkeiten fort, was zu Verzerrungen des Pulses führt.
Man kann im Monomodenfaser folgende Dispersionsarten unterscheiden:
\begin{description}
  \item[Materialdispersion:] Dies ist eine direkte Konsequenz der
    Abhängigkeit des Brechungsindexes von der  Wellenlänge und es geschieht
    nicht nur in optischen Fasern, sondern auch in jedem Glas und ist unabhängig von
    der Geometrie der Faser.
  \item[Wellenleiterdispersion:] Der Lichtpuls propagiert auch im Mantel,
    der einen anderen Brechungsindex als der Kern hat. Wenn man die
    Wellenlänge variiert, gibt es einen Übergang zwischen Propagation,
    die meist im Kern und Propagation, die meist im Mantel stattfindet.
    Dies macht den effektiven Brechungsindex abhängig von der Wellenlänge.
  \item[Profildispersion:] Sie ist kleiner als die Moden- und
    Materialdispersion. Der Unterschied zwischen den Brechungsindexen von Kern
    und Mantel ist auch von der Wellenlänge abhängig.
  \item[Polarisationsmodendispersion:] Sie ist auch klein. Die Polarisation des
    Lichtes kann in zwei orthogonale Komponenten zerlegt werden. Der
    Brechungsindex von Glas ist streng genommen nicht ganz kreissymmetrisch,
    d.h. jedes Glas ist ein bisschen doppelbrechend. Verschiedene
    Brechungsindizes für jede Polarisationskomponente führen zur Dispersion.
\end{description}

\subsection{Nicht-lineare Degradationen}
\paragraph{Vierwellenmischung}
\paragraph{Selbstphasenmodulation}
\paragraph{Kreuzphasenmodulation}

\subsection{Versuch}
Es werden Übertragungsverluste an einem optischen Übertragungssystem
mithilfe von \emph{Optischer Zeitbereichsreflektometrie (OTDR)} und
einem Powermeter untersucht.
Die \emph{Rückstreukurve} soll durch unterschiedliche Fasertypen etc.
manipuliert werden.

\subsubsection{OTDR Einführung}
Mittels OTDR misst man die Dämpfung der Signal im Träger. Vorteile sind, dass
es eine zerstörungsfreie Messung ist, die \emph{in situ} an nur eine der Enden
des Fasers gemacht werden kann. Man bekommt damit Information über die
Abhängigkeit der Dämpfung von der Länge und über die Einfügedämpfung, womit man
inferieren kann, ob es Defekten, Spleißen, Biegungen oder Koppler sich im Faser
befinden, sowie die Lage desselben. Die Ortsauflösung der Messung ist eine
Faltung zwischen die Pulsbreite und die Antwortfunktion des Detektionssytems.  

\subsubsection{Funktionsprinzip} Das OTDR sendet Lichtpulse durch eine Ende des
Fasers, und misst wieviel Zeit passiert ist, bis die Lichtsignale zurückkommen.
Als die Pulse sich im Fiber fortpflanzen, treffen sie streuende und
reflektierende Stellen, was das Licht zurück ins Apparat sendet. Die
physikalische Ursachen davon sind Rayleigh-Streuung und Fresnel-Reflexion.
Indem man die Ankunftszeit dieses zurückkehrendes Licht misst, kann man die
Position und Größe der Fehler im Faser bestimmen.

Ein Pulsgenerator, der von einem digitalen Signalprozessor getriggert wird,
moduliert die Intensität eines Lasers. Typischerweise handelt es sich um ein
viereckiger Laserpuls mit eine Breite von 5 ns bis 10 $\upmu$s. Ein
Richtkoppler leitet das Rückkehrende Signal in einem Photodetektor und wird
dann verstärkt, in einem ADC digitalisiert und im Signalprozessor analysiert.
Die Abtastrate des ADC bestimmt die räumliche Weite der Abtastwerten. Zum
Beispiel, eine Abtastrate von 50 Hz führt zu eine Räumliche Weite von 2 m. Um
die räumliche Weite zu verkleinern ist es in der Praxis günstiger, die
Abtastwerten von verschiedene Messungen zu mischen, als die Abtastfrequenz zu
erhöhen, in einem Prozess, der Interleaving genannt wird.

\subsubsection{Faser Spur}
Nach der Messung bekommt man eine Spur, die das Faser charakterisiert. Der
Graph wird folgendermassen interpretiert:
\begin{description}
  \item[$x$-Axis:] Zeigt die Distanz vom Messapparat der rückstreuende Stelle
    in m.
  \item[$y$-Axis:] Entspricht die Leistung in dB des Rückstreulichtes.
\end{description}
Es wird ein Umrechnungsfaktor von 
\begin{equation*}
1 \ \mathrm m = 10^{-8} \ \mathrm s 
\end{equation*}
benutzt um die Distanz aus der verstrichene Zeit anzugeben. Die
Lichtgeschwindigkeit im Faser ist
\begin{equation*}
  v = \frac{c}{n}, \quad\quad \text{mit } n = 1.5 \text{ für Quarzglas.}
\end{equation*}
Damit erhalten wir eine Geschwindigkeit im Medium von 
\begin{equation*}
  v = \frac{3 \cdot 10^{8}}{\frac{3}{2}} = 2 \cdot 10^{8}
  \ \frac{\mathrm m}{\mathrm s}.
\end{equation*} Das Licht braucht also $\frac{1}{2} 10^{-8}$ Sekunden um 1
Meter sich fortzupflanzen. Aber man misst eigentlich das doppelte Zeit, da man
die Zeit des Echos registriert. Damit ergibt sich der obige Umrechnungsfaktor
von $1\ \mathrm m = 10^{-8} \ \mathrm s$. Die Leistungsverluste am Detektor
müssen auch halbiert werden, denn sie die Verluste der Hin- und Rückweg
entsprechen. Daher werden die Werte im $y$-Axis, die in dB gegeben werden, mit
einem Faktor $5\log_{10}$ anstatt von $10\log_{10}$ berechnet.

Am Faser Spur erkennt man 3 verschiedene Arten von Features:
\begin{description}
  \item[gerade Linien:] Sind von der verteilte Rayleigh Streuung verursacht.
  \item[positive Spizen:] Sind von diskrete Reflexionen verursacht.
  \item[positive oder negative Stufen:] Entsprechen nicht-reflektierende
    Ereignisse. 
\end{description}

\emph{Nicht-reflektierende Ereignisse} deuten mechanische Biegungen des Fasers,
die Verluste durch dem Mantel verursachen, oder Schweiss Spleißen an. Sie
zeigen eine plötzliche Senkung der Rückstreuleistung. Im Fall von Spleißen,
wenn die gespleißten Fasern identisch waren, dann entspricht die Größe der
Stufe, die Einfügedämpfung des Spleißes. Wenn die gespleißte Fasern aber
verschieden waren, dann kann man nur die genaue Einfügedämpfung bekommen, indem
man der Mittelwert über zwei Messungen, je an eine der Enden des Fasers,
berechnet. 

\emph{Reflektierende Ereignisse} sind wegen Unterschiede der Brechungsindex,
die zur Fresnel-Reflexionen führen verursacht. Sie werden als Spitzen im
Rückstreukurve registriert. Diese Reflexionen können von mechanische Spleißen
verursacht werden oder von optische Stecker. In beiden Fällen sind kleine
Gebiete mit Luft im Faser gegeben, und damit ergibt sich beim Glas-Luft
Übergang eine Fresnel-Reflexion.

\subsubsection{Messung}
Im Versuch wurde mit zwei verschiedene Pulsbreiten gemessen, und zwar 50 ns und
100 ns. In Abbildung \ref{fig:pw50} wird die Spur des Fasers dargestellt, die
mit eine Pulsbreite von 50 ns gemessen war. 
\begin{figure}[H]
  \centering
  \begin{tikzpicture}
    \begin{axis}[
	title = OTDR Messung Pulsweite $50\ \mathrm{ns}$,
	xlabel = Entfernung vom Detektor in m,
	ylabel = Leistung am Detektor in dB,
      ]
      \addplot[samples=200] gnuplot[raw gnuplot]{plot 'messungen/OTDR_PW50.dat'};
      \node[pin=45:Anfangsreflexion] at (59, 44760) {};
      \node[pin=90:Spleiß] at (2046, 36524) {};
      \node[pin=90:Spleiß] at (4028, 35827) {};
      \node[pin=90:Stecker] at (5964, 36023) {};
      \node[pin=90:Ende] at (8094, 35294) {};
    \end{axis}
  \end{tikzpicture}
  \caption{Faser Spur gemessen mit eine Pulsbreite von 50 ns. Verschiedene
  erkennbare Ereignisse sind markiert. Es gibt keine Reflexion am Ende, da 
  dies mit Absicht unregelmäßig gebrochen war.}
  \label{fig:pw50}
\end{figure}



\newpage
\section{Komponenten eines faseroptischen Übertragungssystems}
\subsection{Komponenten}
\paragraph{Sendeseite}
\paragraph{Empfängerseite}
\paragraph{Modulator}
\paragraph{Koppler}
\paragraph{optische Verstärker}
\paragraph{Polarisationssteller}
\paragraph{Isolator}
\paragraph{Zirkulator}
\paragraph{Faser-Bragg-Gitter}

\subsection{Versuch}
\subsubsection{Spektren}
\subsubsection{Signalübertragungsverhalten Mach-Zehnder-Modulator}
\subsubsection{Herstellung und Untersuchung Faser-Bragg-Gitter}


\newpage
\section{Systemdesign sowie Multiplex und Modulationsformate}
\subsection{unterschiedliche Multiplextechniken}
\subsection{unterschiedliche Modulationsformate}
\subsection{Parameter für Systemdesign}
\paragraph{Leitungsbudget}
\paragraph{Bitfehlerrate}
\paragraph{Jitter}
\paragraph{Augendiagramm}
\paragraph{Signal-zu-Rausch-Verhältnis}
\paragraph{Empfängerempfindlichkeit}

\subsection{Versuch}
Charakterisierung der optischen Übertragungsstrecke:
Bitfehlerrate in Abhängigkeit von der Dämpfung,
Augendiagramm in Abhängigkeit von der Dämpfung.

\end{document}
